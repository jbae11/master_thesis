
\Cyclus is an agent-based fuel cycle simulation framework 
\cite{huff_fundamental_2016}, which means 
that each reactor, reprocessing plant, and fuel fabrication plant is modeled as an agent.
A \Cyclus simulation contains prototypes, which are fuel cycle facilities with
pre-defined parameters, that are deployed in the simulation as \texttt{facility} agents.
Encapsulating the \texttt{facility} agents are the \texttt{Institution} and \texttt{Region}.
A \texttt{Region} agent holds a set of \texttt{Institution}s. 
An \texttt{Institution} agent can deploy or decommission \texttt{facility} agents.
The \texttt{Institution} agent is part of a \texttt{Region} agent,
which can contain multiple \texttt{Institution} agents. Several versions of \texttt{Institution}
and \texttt{Region} exist, varying in complexity and functions \cite{huff_extensions_2014}.
 \texttt{DeployInst} is used as the institution archetype for this work, where the institution
deploys agents at user-defined timesteps. 

At each timestep (one month),
agents make requests for materials or bid to supply them and exchange
with one another. A market-like mechanism called the dynamic resource exchange
\cite{gidden_agent-based_2015} governs the exchanges.
Each material resource has a quantity, composition, name, and a unique identifier
for output analysis. 

In this work, each nation is represented as a distinct \texttt{Region} agent,
that contains \texttt{Institution} agents, each deploying  \texttt{Facility} 
agents. The \texttt{Institution} agents then deploy agents according to 
a user-defined deployment scheme.

