
Since the scope of this study is broad and interdisciplinary,
I read through numerous literature to gain insight into the
general consensus of advanced fuel cycles and completed studies.


\section{Metrics and Parameters}

An evaluation of a complex system like the nuclear fuel cycle
requires clear, quantitative metrics.
There have been many honest efforts by expert groups
to identify and quantify the key metrics of fuel cycles,
 such as the Evaluation and Screening Study
\cite{wigeland_nuclear_2014}. However, the metrics from the E\&S study
are not quite appropriate for transition scenarios, for the evaluation and simulation
were done in static fuel cycles. This work intends to find important
metrics in a transition scenario.
For reference, the metrics
identified in the Evaluation and Screening study is listed in
table \ref{tab:es}.

\begin{table}[h]
    \centering
    \caption{Metrics considered in Evaluation and Screening
                Study \cite{wigeland_nuclear_2014}.}
    \label{tab:es}
    \begin{tabular}{cc}
        \hline
        \textbf{Category} & \textbf{Parameter} \\ \hline
        \multirow{5}{*}{\shortstack{Nuclear Waste Management \\ (per energy)}} & Mass of \gls{UNF} + \gls{HLW} \\
         & Activity of \gls{UNF} + \gls{HLW} at 100 years \\
         & Activity of \gls{UNF} + \gls{HLW} at 100,000 years \\
         & Mass of \gls{DU} + \gls{RU}\\
         & Volume of \gls{LLW} \\ \hline
        \multirow{2}{*}{\shortstack{Proliferation Risk / \\ Material Security}} & Material Attractiveness \\
         & Targets for Malevolent Use \\ \hline
        \multirow{4}{*}{\shortstack{Environmental Impact \\(per energy)}} & Land use \\
        & Water use \\
        & CO2 released\\
        & Radiological exposure - worker dose\\ \hline

        \multirow{1}{*}{\shortstack{Resource Utilization\\ (per energy)}} & Natural U required \\ \hline

        \multirow{4}{*}{Others} & Safety \\
        & Development and Deployment Risk \\
        & Institutional Issues\\
        & Financial Risk and Economics \\ \hline
    \end{tabular}
\end{table}

While metrics are the outputs of a fuel cycle transition, the parameters
are inputs to a fuel cycle transition scenario, and are subject to
variability. The varying parameters impacts the metrics, which causes
the uncertainty in the outcome of a fuel cycle.
Key parameters in fuel cycle transitions have been identified
by the OECD-NEA study on uncertainty \cite{hyland_effects_2015}. In this
study, some of these parameters (organized in table \ref{tab:sens})
are swept over a range to quantify its effect on the metrics.
As mentioned in the introduction, the correlations are with static
metrics, not what parameters are important for incentivizing transition.

\begin{table}[h]
    \centering
    \label{tab:sens}
    \caption{Parameters considered in uncertainties study
             in OECD-NEA study \cite{hyland_effects_2015}.}
    \begin{tabular}{cc}
        \hline
        \textbf{Category} & \textbf{Parameter} \\ \hline
        \multirow{4}{*}{Fissile Inventory and Consumption} & Fissile Burnup \\
         & Fresh fuel composition \\
         & Cycle length \\
         & Breeding Grain\\ \hline
        \multirow{4}{*}{Waste} & Initial \gls{MA} content in fuel \\
         & Recuperation Rate (MA) \\
         & Reprocessing Efficiency \\
         & Enrichment Tail \\ \hline
        \multirow{4}{*}{Fabrication and Reprocessing} & Minimum cooling time \\
        & Fabrication Time\\
        & Reprocessing begin time\\
        & Reprocessing capacity\\ \hline
        \multirow{4}{*}{General} & Reactor Lifetime\\
        & Nuclear Energy Demand \\
        & FR introduction time \\
        & Rate of transition \\ \hline
    \end{tabular}
\end{table}

Publications from MIT, while taking a more
reluctant stance on the need to transition into a `closed' fuel cycle,
provides broader metrics like economics,
safety, waste management, or energy independence.
 The report's criteria to compare fuel cycles
is replicated in table \ref{tab:mit}.

\begin{table}[h]
    \centering
    \label{tab:mit}
    \caption{Criteria to Compare Fuel Cycles from the MIT report \cite{kazimi_future_2011}}
    \begin{tabular}{ccc}
        \hline
        \textbf{Criteria} & \textbf{Technical} & \textbf{Institutional}\\ \hline
        Economics & Overnight Capital Costs & Financing, Regulation \\
        Safety & Risk Assessment & Regulatory structure \\
        Waste Management & Waste form, time of storage & Regulation, Societal view \\
        Environment & Water / land consumption & Greenhouse gas regulation \\
        Resource utilization & Uranium costs & Security of supply \\
        Nonproliferation & Separated plutonium & Institutional arrangements for fuel \\
        \hline
    \end{tabular}
\end{table}

\subsection{Proliferation resistance}

Proliferation resistance of a nuclear material is a difficult
metric to quantify,
for the concept combines material attractiveness (easier to weaponize)
and its intrinsic and extrinsic barriers (activity, state of matter).
Moreover, defining a proliferation resistance metric for an
entire fuel cycle is much more difficult, since materials change
compositions and forms throughout. However, general consensus exists
that having less separated streams of special nuclear material, such
as plutonium and \gls{HEU} increases proliferation resistance.
Charlton et al. \cite{charlton_proliferation_2007} suggests a
quantitative, time-varying, multi-attribute utility analysis method
to assess the proliferation resistance of a fuel cycle. The metric
depends on attractiveness level, concentration, handling requirements,
type of accounting system, and accessibility. The paper also analyzes
metric throughout a single fuel cycle, and shows that the proliferation
resistance changes over time in a fuel cycle.

This paper demonstrates that an analysis of a fuel cycle requires
a dynamic approach to measuring its metrics, and a similar approach
can be used to measure the metrics in a transition scenario.
For example, reprocessing \gls{UNF} that are currently in the
pools might have a negative effect on proliferation resistance
since it produces separated plutonium streams. However, it reduces
the \gls{UNF} inventory, which increases proliferation resistance.
Thus, the focus should be on analyzing how the metrics change
over time during transition and after equilibrium, and what
incentives (or lack thereof) there are for `closing' the fuel cycle.

\section{Economics}
Currently, most fuel cycles in the world are once-through,
with the exception of France's \gls{MOX} reprocessing for its
\glspl{LWR}. Because there has not been a commercial advanced
fuel cycle, with large-scale reprocessing and fast reactor
operation,there are large cost uncertainties are associated
with fuel cycle strategies \cite{d._e._shropshire_advanced_2009}.

Also, with the now abundant uranium resources and a possibility
of uranium seawater extraction \cite{tabushi_extraction_1979},
prospective uranium fuel costs are relatively
low. This led to a general consensus that `closed' cycles are
more expensive than once-through-cycles
\cite{d._e._shropshire_advanced_2009, bunn_economics_2005, charpin_economic_2000}.

With a high risk and uncertainty associated with advanced fuel cycles,
and the fact that taking the risk to transition may cost more,
economics does not seem to be a dominant incentive for transitioning
into an advanced fuel cycle.

\section{Sustainability}
Nuclear energy generation has repeatedly gained criticism due to its
sense of not being sustainable - notably to the fact that uranium
supply is limited and that the waste is radioactive
for a very long time \cite{dittmar_nuclear_2012}. Even with a permanent repository sited,
which is not the case, it would be difficult to gain support of the
public to rely on a limited system that requires active attention.
 The current once-through
fuel cycle does not optimize the consumption of a natural resource
and utilize energetic materials like the actinides, which is far
from achieving full sustainability \cite{poinssot_recycling_2012}.
Thus, the focus on sustainability of nuclear energy should be
to maximize the resource utilization (minimizing natural resource
input) and reduce the footprint of nuclear energy production
(reduction of waste) \cite{poinssot_assessment_2014}.

A `closed' fuel cycle can achieve this goal in two ways -
recycling \gls{LWR} \gls{UNF}, and burning of \gls{MA}.
For example, recycling \gls{LWR} \gls{UNF} to provide
fuel for advanced breed-and-burn reactors increases
the uranium utilization by a factor of three relative
to the once-through cycle \cite{zhang_improved_2018}.
By burning the \gls{MA}, which have long half-lives,
the acute radio-toxicity time of \gls{UNF} can be reduced
from millennia to centuries \cite{nash_1-_2015}, since the
\glspl{MA} account for most of the long-term radio-toxicity,
while the major fission products decay off in the first 500
years \cite{hardin_thermal_2011}.

\section{Energy Security}
The \gls{IEA} defines energy security as the uninterrupted
availability of energy sources at an affordable price, with
its multiple aspects such as timely investments to supply
energy in line with economic developments and environmental
needs \cite{lefevre-marton_energy_2005}. Since I am looking
at one source of energy, nuclear, the energy portfolio policy
approach \cite{winzer_conceptualizing_2012,
lefevre-marton_energy_2005} is not valid for this study.
One of the suggested indicators of energy security is reserves-to-production
\cite{feygin_oil_2004}, which is used for oil. Since advanced
fuel cycles produce fuels along with its consumption, this
value can be translated to reserves-to-net-consumption, which is
the ratio between the fuel (or fissile) inventory to its
net consumption every year.

Also, many countries focus on the fuel security aspect
of energy security when considering nuclear. Since enriched
\gls{UOX} fuel requires multiple stages of processing
(milling, conversion, enrichment and fabrication),
the fuel supply chain is risky and can be prone to delays
\cite{association_ensuring_2006}. Advanced reactors can
reduce this risk by producing its own fuel (given
that there is in-house reprocessing, like an \gls{MSR}),
and reducing the amount of input into the reactor.
