
To quantify the important metrics for incentives to transition
into a closed fuel cycle, I take the following steps:

\begin{enumerate}
    \item Benchmark \Cyclus against ORION and \gls{UNF-STANDARDS}
    \item Identify key fuel cycle transition metrics
    \item Identify key fuel cycle parameters
    \item Perform parameter sweep on key parameters on real scenarios
    \begin{enumerate}
        \item United States transition scenario
        \item France transition scenario
    \end{enumerate}
    \item Analyze sensitivity of parameters on key metrics
    \item Visualization and discussion
\end{enumerate}

\subsection{Material Flow}

The fuel cycle is represented by a series of facility agents whose material 
flow is illustrated in figure \ref{diag:fc}, along with
the \Cyclus archetypes that were used to model each facility.
In this diagram, \gls{MOX} Reactors include both French \glspl{PWR} and 
\glspl{SFR}.

% Define block styles
\tikzstyle{decision} = [diamond, draw, fill=blue!20, 
text width=4.5em, text badly centered, node distance=3cm, inner sep=0pt]
\tikzstyle{block} = [rectangle, draw, fill=blue!20, 
text width=5em, text centered, rounded corners, minimum height=4em]
\tikzstyle{line} = [draw, -latex']
\tikzstyle{cloud} = [draw, ellipse,fill=red!20, node distance=3cm,
minimum height=2em]


\begin{figure}
        \centering
        \scalebox{0.6}{
                \begin{tikzpicture}[align=center, node distance = 3cm and 3cm, auto]
                % Place nodes
                \node [block] (sr) {Mine (\texttt{SOURCE})};
                \node [cloud, below of=sr] (nu) {Nat U};
                \node [block, below of=nu] (enr) {Enrichment ({\footnotesize \texttt{ENRICHMENT}})};
                \node [cloud, below of=enr] (uox) {\acrshort{UOX}};
                \node [block, below of=uox] (lwr) {\gls{LWR} (\texttt{REACTOR})};
                \node [cloud, right of=lwr] (snf) {\gls{UNF}};
                \node [block, right of=snf] (pool) {Pool (\texttt{Storage})};
                \node [cloud, left of=lwr] (tl2) {Dep U};
                \node [cloud, right of=enr] (tl) {Dep U};
                \node [block, right of=tl] (sk) {Repository (\texttt{SINK})};
                \node [cloud, below of=sk] (cunf) {Cooled \gls{UNF}};
                \node [cloud, below of=pool] (cunf2) {Cooled \gls{UNF}};
                \node [block, below of=snf] (rep) {{\small Reprocessing ({\footnotesize \texttt{SEPARATIONS}})}};
                \node [cloud, below of=rep] (u) {Sep. U} ;
                \node [cloud, left of=rep] (pu) {Sep. Pu};
                \node [block, left of=pu] (mix) {Fabrication (\texttt{MIXER})};
                \node [cloud, below of=mix] (mox) {\gls{MOX}};
                \node [block, below of=mox] (mxr) {\gls{MOX} Reactors 
                (\texttt{REACTOR})};
                \node [cloud, right of= mxr] (snmox) {Spent \gls{MOX}};
                
                \draw[->, thick] (sr) -- (nu);
                \draw[->, thick] (nu) -- (enr);
                \draw[->, thick] (enr) -- (tl);
                \draw[->, thick] (enr) -- (tl2);
                \draw[->, thick] (tl) -- (sk);
                \draw[->, thick] (tl2) -- (mix);
                \draw[->, thick] (enr) -- (uox);
                \draw[->, thick] (uox) -- (lwr);
                \draw[->, thick] (lwr) -- (snf);
                
                \draw[->, thick] (lwr) -- (snf);
                \draw[->, thick] (snf) -- (pool);
                \draw[->, thick] (pool) -- (cunf);
                \draw[->, thick] (pool) -- (cunf2);
                \draw[->, thick] (cunf) -- (sk);
                \draw[->, thick] (cunf2) -- (rep);
                
                \draw[->, thick] (rep) -- (u);
                \draw[->, thick] (rep) -- (pu);
                \draw[->, thick] (pu) -- (mix);
                \draw[->, thick] (mix) -- (mox);
                \draw[->, thick] (mox) -- (mxr);
                \draw[->, thick] (mxr) -- (snmox);
                \draw[->, thick] (snmox) -- (rep);
                
                \end{tikzpicture}
        
                }
                \caption{Fuel cycle facilities (blue boxes) represented by 
                        \Cyclus archetypes (in parentheses) pass materials (red 
                ovals) around the simulation.} 
                \label{diag:fc}
\end{figure}

A mine facility provides natural uranium, which is enriched by an enrichment
facility to produce \gls{UOX}. Enrichment wastes (tails) are disposed of to a 
sink facility representing ultimate disposal. The enriched \gls{UOX} fuels
the \glspl{LWR} which in turn produce spent \gls{UOX}. The used fuel
is sent to a wet storage facility for a minimum of 72 months. \cite{carre_overview_2009}.

The cooled fuel is then reprocessed to separate plutonium and uranium,
or sent to the repository.
The plutonium mixed with depleted uranium (tails) makes \gls{MOX} (Both for
French \glspl{LWR} and \glspl{ASTRID}).
Reprocessed uranium is unused and stockpiled. Uranium is reprocessed
in order to separate the raffinate (minor actinides and fission products)
from usable material. Though neglected in this work, reprocessed
uranium may substitute depleted uranium for \gls{MOX} production. In the
simulations, sufficient depleted uranium existed that the complication of
preparing reprocessed uranium for incorporation into reactor fuel
was not included. However, further in the future where the depleted
uranium inventory drains, reprocessed uranium (or, natural uranium) will need to be utilized. 
