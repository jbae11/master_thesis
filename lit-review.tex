\subsection{Benchmark Study of Computer Codes for System Analysis of the Nuclear Fuel Cycle}
\cite{benchmark_laurent_2009}

Important fuel cycle simulation parameters:
Discrete batches (only COSI6) allows more complex comparison of results
Decay of nuclides
depletion calculation


\subsection{The Future of the Nuclear Fuel Cycle}
\cite{MITEI}

Better focus on LWR technology, since the uranium resources
won't be a problem for another century or so.

Comments on obvious things like we need a long term unf repository
where a consent-based, local community cooperative strategy is employed.

\textbf{"Closed fuel cycle design has focused on what goes back to the reactor but not on how wastes are managed. "}

US classifies waste depending on source, not hazard.

Fuel leasing from big countries to small countries
to provide an incentive to not pursue nuclear weapons program

R\&D Priorities:
better LWR technolgoy
waste disposal options
modeling trade off among options
other nuclear reactor applications (process heat)

\subsection{Benchmark Study on Nuclear Fuel Cycle Transition Scenarios Analysis Codes}

Code Capabilities:
\begin{enumerate}
    \item Front end fuel cycle facilities
    \item Reprocessing Plants
    \item Reprocessing capacity deployed
    \item Spent fuel to be reprocessed
    \item Fissile material availability forecast
    \item User parameter for deployment of reactors
    \item Waste-radioactivity
    \item Waste-decay heat
    \item Waste-radiotoxicity
    \item Waste conditioning modeling
    \item Repository requirement assessment
    \item LLW Modeling
    \item Economics assessment module
    \item Economics optimization
    \item U Price model
    \item Transportation costs
    \item Proliferation metrics
\end{enumerate}

Benchmark:
\begin{enumerate}
    \item Depletion of UOX, MOX, MOX Na-FR fuel (pg 57)
    \item Transition Scenario
\end{enumerate}

Transition Scenarios
Open Cycle / monorecycling of pu in PWR / monorecycling of pu in PWR and
deployment of GEN IV fast reactors recycling pu and MA.

120 years
60 GWe installed power
constant 430 TWhe (load factor: 0.8176)
Variation rate for every type +-2 GWe/year



\subsection{Benefits and concerns of a closed fuel cycle}
\cite{widder_benefits_2010}


\subsection{Advanced Fuel Cycle Economic Analysis of Symbiotic Light-Water Reactor and Fast Burner Reactor Systems}
\cite{shropshire_advanced_2009}

Large cost uncertainties associated with all fuel cycle strategies.

Reveals that a closed fuel cycle are about 10\% more expensive in terms of 
electricity generation cost.


\subsection{Nuclear Fuel Cycle Transition Scenario Studies}
\cite{oecd_nuclear_2009}

Key issues:
\begin{enumerate}
    \item Time lines: Implementation and development of required technologies
    \item Materials inventory: Storage capacities
    \item Materials management: appropriate inventory needs to exist
    \item Material dynamics impact: Dynamics effects
    \item Economic: Cost should be low
\end{enumerate}


Metrics (National Energy policy objectives and associated technology requirements):
\begin{enumerate}
    \item Enhance proliferation resistance
    \item Waste management and disposal
    \item Reduce HLW repository
    \item Minimize environmental impact
    \item Security of energy supply
\end{enumerate}


Externalities:
Internalisation of external costs -> security of supply / carbon emissions
\textbf{Value of actinide burning -> no more long-term stewardship of waste}


\subsection{Proliferation Resistance Assessment Methodology for Nuclear Fuel Cycles}
\cite{charlton_proliferation_2007}

Proliferation resistance is a measure of the relative increase in barriers 
(intrinsic to material and extrinsic) to impede the proliferation of nuclear weapons
either by diversion of material by a state in possession of a system of theft
of material by a terrorist sub-national group.

Commonly agreed upon attributes for proliferation resistance:
\begin{enumerate}
    \item Extraordinary reduction of special nuclear material (Pu, HEU)
    \item Avoidance of separated SNM streams
    \item Designing material / process so that it can be more readily safeguarded (material accountancy and containment / surveillance)
\end{enumerate}

Focus should be on material moving through the fuel cycle

Tracks a unit mass of material through its fuel cycle and assesses the proliferation resistance

total nuclear security metric

\textbf{Extremely reproducible metric evaluation}


\subsection{The Economics of Reprocessing Versus Direct Disposal of Spent Nuclear Fuel}
\cite{bunn_economcis_2004}

General agreement that reprocessing and recycling is more expensive than direct
disposal of spent fuel \cite{charpin_economic_2000}


\subsection{Thoughts}

It seems like the decision to invest and implement advanced fuel cycles and 
reactors depend heavily on the future prospect of nuclear energy demand.

If a market develops for other side products (like hydrogen, desalinization),
there might be a bigger economic incentive to develop advanced reactors and 
fuel cycles.


