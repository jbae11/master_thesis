Nuclear energy provides the largest amount of carbon-free energy
in the United States. However, several problems exist with nuclear
energy generation, most notably, the used nuclear fuel accumulating
in the United States.
From the 1950s, the United States has been trying to commission a
geological waste disposal site for its `spent' fuel. However, even
after 60 years, this endeavor has yet to accomplish of practical value.

Recent advances in nuclear technology produced
a multitude of new reactor designs which experiment with new forms
of fuel that deviates from an enriched uranium oxide fuel.
Noble designs such as fast reactors and molten salt reactors
allow various types of fuel forms and fuel cycles to be considered. These advances in new
technology fueled the dream of a `closed' fuel cycle, where little to no
supply of natural material is required, that `used' fuel from one reactor,
after processing, could fuel another reactor.

However, transitioning the nuclear fuel cycle requires great investment,
from constructing new reactors to new fuel processing infrastructure.
Thus, it seems highly unlikely that, given the current circumstances
such as the economic competitiveness of nuclear, decrease in the price of natural
gas, an administration unenthusiastic towards sustainability, that
this transition will happen soon.

In 2017, \gls{ANS} announced closing the fuel cycle as one of its
grand challenge, stating that doing so will allow the U.S.
to `obtain maximum value and minimizing environmental
impact from using nuclear fuel' \cite{_ans_2017}.
The report also says that identifying the `most promising'
fuel cycles based on clear evaluation metrics is crucial towards
this effort, as it was done in the Fuel Cycle Evaluation and Screening
Study report \cite{wigeland_nuclear_2014}. The Study concludes that
a fuel cycle with continuous reprocessing of the \gls{UNF} to
fuel fast reactors is the most promising fuel cycle.

However, the Fuel Cycle Evaluation and Screening Study report
evaluates fuel cycles in static conditions rather than
in a dynamic fuel cycle transition scenario.
Thus, I question the important metrics when considering
a nuclear fuel cycle transition - what will persuade
the public to support such a great effort? What incentives
do we have to `close' the fuel cycle?

Additionally, we need to identify major parameters in
fuel cycle transition, so that resources can be
focused on the more important parameters, such as reprocessing
efficiency or advanced reactor technology readiness.

These series of studies attempt to answer those questions with a
combination of literature review, fuel cycle simulations, and sensitivity
studies.
