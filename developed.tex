I developed two extensions to leverage 
the capabilities of Cyclus to model
real-world fuel cycle transition scenarios.
The first extension is a python input-generating module
that automates scenario generation of the real-world 
nuclear fleet at any point in time. The second extension
is a Cyclus archetype that mimics \gls{MSR} feed and removal
behavior using an HDF5 database generated from SaltProc.

\section{cyclus-input-gen}
\label{sec:writeinput}
The python package \texttt{cyclus-input-gen} contains
multiple submodules for automating Cyclus input generation.
The submodule \texttt{from\_pris} automates the generation of Cyclus input files
to model the state of reactor fleets at a given
point in time.

The script reads from the \gls{PRIS} database \cite{iaea_nuclear_2018}
and extracts data on each reactor's country, reactor unit,
type, net capacity (MWe), status, operator, construction date,
first criticality date, first grid date, commercial date,
and shutdown date (if applicable). The user inputs simulation
configurations such as start year, start month, and
simulation duration. The module uses the collected data to
fill out a template into a Cyclus input file. The logical flow of the module is 
shown in diagram \ref{diag:log_flow}.


\begin{figure}
	\centering
	\begin{tikzpicture}[node distance=3.5cm and 2cm, on grid, auto, text width=2cm]
	\node (database) [object] {PRIS Database (\texttt{.csv})};
	\node (data) [process, below of=database] {Reactor data (net capacity, reactor size)};
	
	\node (user) [object, right of=database] {User};
	\node (input) [process, below of=user] {Start year, month,  Simulation duration};
	
	\node (template) [process, right of=user] {Cyclus input template files};
	
	\node (script) [object, below of=input] {\texttt{from\_pris}};
	
	\node (lifetime) [process, below of=script] {Calculated reactor lifetime};
	\node (coresize) [process, right of=lifetime] {Interpolated core size};
	\node (region) [process, left of=lifetime] {Calculated deployment scheme};
	
	\node (output) [object, below of=lifetime] {Generated Cyclue Input File (.xml)};
	
	\draw [arrow] (database) -- (data); 
	\draw [arrow] (user) -- (input); 
	\draw [arrow] (data) -- (script);
	\draw [arrow] (input) -- (script);
	\draw [arrow] (template) -- (script);
	
	\draw [arrow] (script) -- (lifetime);
	\draw [arrow] (script) -- (coresize);
	\draw [arrow] (script) -- (region);
	
	\draw [arrow] (lifetime) -- (output);
	\draw [arrow] (coresize) -- (output);
	\draw [arrow] (region) -- (output);
	
	\end{tikzpicture}
	\caption{ Logic flow of \texttt{from\_pris}.
		Green circles and blue boxes represent files and data, respectively.}
	\label{diag:log_flow}
\end{figure}

\subsection{Reactor deployment calculation}
The module calculates the deployment
scheme of reactors and their lifetimes by assuming
that all reactors shut down after 60 years of operation.
If the expected shutdown date is later than the user-input
simulation start date,
the reactor is not written in the input. If the reactor was
operational prior to the simulation start date,
and its shutdown date later than simulation start date,
the reactor is deployed at the beginning of simulation with its remaining lifetime.
If the reactor's start date is later than the simulation
start date, and the shutdown time is undefined, the reactor
is deployed at the defined start date with 60 years of lifetime.

\subsection{Reactor parameter calculation}

The module calculates the core sizes of various reactor
types by using a linear core size model. It assumes that
the number of assemblies in a 
reactor core scales linearly from a model reactor design,
as shown in the equation below. The mass per assembly
is kept constant.
The model reactor designs are listed in table \ref{tab:lin_core}.

\[
N_{assem} = N_{assem.ref} * \frac{\text{P}}{P_{ref}}
\]
\[
N_{assem.pwr} = 157 * \frac{\text{P}}{1,100}
\]
\[
N_{assem.bwr} = 764 * \frac{\text{P}}{10,098}
\]
\[
N_{assem.phwr} = 4,560 * \frac{\text{P}}{700}
\]
\[
P = \text{power capacity of reactor}
\]

\begin{table}[h]
	\centering
	\caption{Reactor model designs used for the linear core size model.}
	\begin{tabularx}{\textwidth}{bsssss}
		\hline
		\textbf{Category} & \textbf{Model Reactor} & \textbf{Power [MWe]} & \textbf{Assembly Mass [kg]} & \textbf{Assemblies in Core} & \textbf{Reference}  \\
		\hline
		\gls{PWR} & AP-1000 & 1,110 & 446 & 157 & \cite{schulz_westinghouse_2006} \\
		\gls{BWR} & 4-MK I & 1,098 &180 & 764 & \cite{moore_physical_1989} \\
		\gls{PHWR} & CANDU6 & 700 & 24.17 & 4,560 & \cite{galeriu_technical_nodate} \\
		\hline
	\end{tabularx}
	\label{tab:lin_core}
\end {table}



\section{SaltProc-reactor}
The SaltProc reactor is a Cyclus facility archetype designed to 
model \gls{MSR} behavior using a database. It roughly couples ``SaltProc'' \cite{rykhlevskii_online_2017}
and Cyclus, by using the output from SaltProc to mimic \gls{MSR}
feed and removal behavior in Cyclus. Most of the computationally heavy
work (neutron transport calculations, fuel depletion calculation) is done in SERPENT (driven  by SaltProc)
in generating the database, which avoids the large computational burden
when running Cyclus. 

This method is similar to the simplified implementation of
recipe reactors, where the depletion calculation is performed outside of the
fuel cycle simulation. Instead of a single depletion calculation
used in a recipe reactor, this reactor uses a database of recipes
to capture the continuously varying state of liquid-fueled reactors like \glspl{MSR}. 


\begin{figure}
	\centering
	\begin{tikzpicture}[node distance=3.5cm and 2cm, on grid, auto, text width=2cm]
	
	\node (msr) [object] {\gls{MSR} design};
	\node (repro) [object, right of = msr] {Reprocessing Scheme};
	\node (timesteps) [object, right of = repro] {Timesteps};
	
	\node (saltproc) [process, below of = repro] {SaltProc};
	\node (serpent) [process, right of = saltproc] {SERPENT 2};
	
	\node (hdf5) [object, below of = saltproc] {HDF5 Database};
	
	\node (hdf5 reactor) [process, below of = hdf5] {HDF5-reactor};
	
	\node (bid) [object, below of = hdf5 reactor] {Feed Material request};
	\node (request) [object, right of = bid] {Waste Material bid};
	\node (power) [object, left of = bid] {Power};
	
	\node (cyclus) [process, below of = bid] {Cyclus};


	\draw [arrow] (msr) -- (saltproc); 
	\draw [arrow] (repro) -- (saltproc); 
	\draw [arrow] (timesteps) -- (saltproc);
	
	\draw [arrow] (saltproc) -- (serpent);
	\draw [arrow] (serpent) -- (saltproc);
		
	\draw [arrow] (saltproc) -- (hdf5);
	\draw [arrow] (hdf5) -- (hdf5 reactor);
	
	\draw [arrow] (hdf5 reactor) -- (bid);
	\draw [arrow] (hdf5 reactor) -- (request);
	\draw [arrow] (hdf5 reactor) -- (power);
	
	\draw [arrow] (bid) -- (cyclus);
	\draw [arrow] (request) -- (cyclus);
	\draw [arrow] (power) -- (cyclus);
	
	
	\end{tikzpicture}
	\caption{ Logic flow of \texttt{HDF5 Reactor}.
		Green circles and blue boxes represent files and data, respectively.}
	\label{diag:hdf5_flow}
\end{figure}

\subsection{Code Description}
The user provides only of the commodity names
for each stream (e.g. waste, fertile), and the database path, since the
HDF5 database already contains the notion of reactor design,
reprocessing scheme, and other reactor parameters (shown in
figure \ref{diag:hdf5_flow}). The commodity names are needed
for reactor agents to communicate with other Cyclus agents
in exchanging material.

At every timestep, 
The \texttt{HDF5 Reactor} calculates the material mass
and composition accumulated during the Cyclus timestep,
as shown in equation below.

\[
M_{T} = \sum m_T
\]
\[
m_{T} = \sum_{t=(T-1)}^{T} m_{t}
\]
\[
M_T = \text{total mass of stream in one Cyclus timestep}
\]
\[
m_T = \text{mass of isotope in one Cyclus timestep}
\]
\[
m_t = \text{mass of isotope in one SaltProc timestep}
\]


\section{Limitations of the database approach}
\label{sec:limit}
The limitations of this database approach is that it does not
take into account the changing incoming fuel compositions due to decay.
The separated \gls{TRU} composition may vary depending on the time
an \gls{LWR} \gls{UNF} has been cooled, thus affecting the performance
of the \gls{MSR}. The database approach assumes a fixed input salt
composition, which is not the case in this simulation, because reprocessed
\gls{TRU} spends varying amounts of time until it is fabricated and
put in the \gls{MSR}.

This limitation is similar to that of batch-wise recipe reactors, where the 
pre-generated recipe already has a notion of the composition of initial fuel.
This results in a depleted fuel composition that is agnostic to the incoming
fuel composition in the simulation. For example, a batch-wise recipe reactor
depleting a \gls{MOX} fuel would deplete the \gls{MOX} fuel to a same
composition regardless of its plutonium vector. This is mediated by using
fuel fabrication facility that modifies the plutonium enrichment to
meet a certain fissile value (in Cyclus, the Cycamore \texttt{FuelFab} archetype).

The same method can be applied, where, instead of using a fixed mass ratio
to fabricate fuel for \glspl{MSR}, the fuel fabrication is done by modifying
the \gls{TRU} enrichment to match a certain fissile value. However, this is
not a solution to solve the accuracy problem for depletion and errors
in neutronics calculations.

Another way to overcome this limitation is to use a set of databases 
that contain multiple SaltProc simulation results with varying initial \gls{TRU}
vectors. When the reactor archetype receives the fuel, it will find the
SaltProc simulation result with the initial fuel composition closest to that
of the received fuel salt, minimizing error stemming from varying
\gls{TRU} compositions.

The best solution is to have a built-in depletion calculation model. However,
as mentioned before, this requires too much computational burden for a fuel cycle simulation,
where there are multiple reactors `at play' at any given time. Another
possible method is to implement a reduced-order-model of \gls{MSR} depletion
behavior created by training from a large dataset of \gls{MSR} depletion
calculations. 