I developed two extensions to leverage 
the capabilities of Cyclus to model
real-world fuel cycle transition scenarios.
The first extension is a python input-generating module
that allows automated scenario generation of the real-world 
nuclear fleet at any point in time. The second extension
is a Cyclus archetype that mimics \gls{MSR} feed and removal
behavior using a HDF5 database generated from SaltProc.

\section{write\_input.py}
The objectives for the \texttt{write\_input.py} module
is to automate the population of Cyclus input files
to model the current state of reactor fleets at a given
point in time. 

The module reads from the \gls{PRIS} database
and extracts data on each reactor's country, reactor unit,
type, net capacity (MWe), status, operator, construction date,
first criticality date, first grid date, commercial date,
and shutdown date (if applicable). The user inputs simulation
configurations such as start year, start month, and
simulation duration. The module then calculates the deployment
scheme of each reactor, and interpolates the core size using
a linear core size model. The linear core size model assumes
that the core and assembly size of an \gls{LWR} scales linearly
with its net power capacity. The logic flow of the module is 
shown in diagram \ref{diag:log_flow}


\begin{figure}
	\centering
	\begin{tikzpicture}[node distance=3.5cm and 2cm, on grid, auto, text width=2cm]
	\node (database) [object] {PRIS Database (\texttt{.csv})};
	\node (data) [process, below of=database] {Reactor data (net capacity, reactor size)};
	
	\node (user) [object, right of=database] {User};
	\node (input) [process, below of=user] {Start year, month,  Simulation duration};
	
	\node (template) [process, right of=user] {Cyclus input template files};
	
	\node (script) [object, below of=input] {write\_input.py};
	
	\node (lifetime) [process, below of=script] {Calculated reactor lifetime};
	\node (coresize) [process, right of=lifetime] {Interpolated core size};
	\node (region) [process, left of=lifetime] {Calculated deployment scheme};
	
	\node (output) [object, below of=lifetime] {Generated Cyclue Input File (.xml)};
	
	\draw [arrow] (database) -- (data); 
	\draw [arrow] (user) -- (input); 
	\draw [arrow] (data) -- (script);
	\draw [arrow] (input) -- (script);
	\draw [arrow] (template) -- (script);
	
	\draw [arrow] (script) -- (lifetime);
	\draw [arrow] (script) -- (coresize);
	\draw [arrow] (script) -- (region);
	
	\draw [arrow] (lifetime) -- (output);
	\draw [arrow] (coresize) -- (output);
	\draw [arrow] (region) -- (output);
	
	\end{tikzpicture}
	\caption{ Logic flow of \texttt{write\_input.py}.
		Green circles and blue boxes represent data and sources, respectively.}
	\label{diag:log_flow}
\end{figure}



\section{HDF5-reactor}
The HDF5 reactor is a Cyclus facility archetype designed to 
model \gls{MSR} behavior. It roughly couples Saltproc \cite{rykhlevskii_online_2017}
and Cyclus, by using the output from SaltProc to mimic \gls{MSR}
feed and removal behavior in Cyclus. Most of the computationally heavy
work is done in SaltProc in generation of the database, which avoids the large computational burden
when running Cyclus. This method is similar to the simplified implementation of
recipe reactors, where the depleted composition of the fuel is calculated outside of the
fuel cycle simulation, but for a reactor with a continuous material feed and removal,
such as an \gls{MSR}.

[logic flow]
[example problem?]

