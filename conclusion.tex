This thesis expands and demonstrates the capabilities of \Cyclus, the agent-based
fuel cycle simulator, to model real-world nuclear fuel cycle
transition scenarios. \Cyclus has a novel framework that is
modular and expandable, that allows addition of functionalities
without editing \Cyclus itself.

I developed \texttt{cyclus\_input\_gen}
to integrate historical nuclear operations by generating
a \Cyclus input file from a \gls{PRIS} database. I also
developed an \gls{MSR} module in \Cyclus that uses a database
generated by a high-fidelity \gls{MSR} simulation to model
\glspl{MSR} in a large fuel cycle simulation. The two
added capabilities leverage \Cyclus' existing capabilities
to model complex, real-world nuclear fuel cycle transition
scenarios.

I compared \Cyclus' results with
the results from other fuel cycle simulators
(
DYMOND \cite{yacout_modeling_2005},
VISION \cite{jacobson_verifiable_2010},
ORION \cite{gregg_analysis_2012}, and
MARKAL \cite{shay_epa_2006}
) for a generic \gls{NFC} transition scenario from an
\gls{LWR} fleet to an \gls{SFR} fleet. Results show
excellent agreement, except for disagreement caused 
by differences in reactor depletion
behavior at the end of reactor lifetime, and
whether fuel discharge is discrete or continuous.
The \Cyclus simulation's \gls{SFR} core size was
$1.08\%$ because \Cyclus can only have integer
number batches, while the benchmark had 3.96
batches, which is unrealistic. This difference
causes the unused \gls{TRU} inventory in the \Cyclus
simulation to be smaller initially, since more
\gls{TRU} is used for the initial loading of the \gls{SFR}
cores. 

I simulated the French \gls{NFC} transition simulation scenario
to demonstrate
Cyclus' capacity to model the historical nuclear operation
of multiple regions. It also showed that
France can transition from its \gls{LWR} fleet into a fully \gls{SFR} fleet
without additional construction of \glspl{LWR} if France receives
\gls{LWR} \gls{UNF} from other \gls{EU} nations.

I simulated the U.S. \gls{NFC} transition simulation scenario
to demonstrate
Cyclus' capacity to roughly model \glspl{MSR} in
a large \gls{NFC} simulation. The simulation showed
that the U.S. can transition into a fully
\gls{MSR} fleet, and that availability of nuclear materials
(\gls{LWR} \gls{UNF} and depleted uranium) is not a constraint.



\section{Future work}
Continued research into better methods of modeling fuel cycle
transition scenarios and fuel cycle facilities can progress
in multiple directions. 

First, efforts can be made to incorporate
uncertainties in future fuel cycle simulations by enabling
\Cyclus to accept statistical distribution and symbolic
functions as input parameters. Parameters such as \gls{LWR} fuel
burnup and enrichment evolve in time and can be better modeled
by a function of time than a static parameter. Similarly, reactor parameters
such as refueling time and cycle time can be better described by sampling
from a distribution (e.g. gaussian) rather than the current static
value.

Second, implementing reactor archetypes that directly
preform depletion calculations could increase material
inventory accuracy.
The reactor archetypes used in this work use spent fuel
recipes depleted outside of \Cyclus by higher fidelity tools,
meaning that there are no depletion calculations performed by
\Cyclus during runtime.
While using a recipe to obtain depleted fuel composition is quick,
it does not take into account the variations in input fuel composition.
Unfortunately, conducting unique depletion calculations for each reactor
discharge represents a significant computational burden.
Additionally, imprecision
introduced through simplifying assumptions inherent in fuel
cycle modeling (e.g. deployment schedules, separation efficiency,
constant reactor cycle time) significantly impact fuel cycle metrics.
These effects typically dwarf precision improvements that higher
fidelity depletion modeling might enable.

\section{Closing remarks}
The \gls{ANS} chose closing the \gls{NFC} as one of its grand challenges. \gls{ANS}
defined the challenge to be solved by firmly establishing the pathway that leads to closing
the nuclear fuel cycle to support the demonstration and deployment of
advanced fission reactors, accelerators, and material recycling technologies
to obtain maximum value while minimizing environmental impact 
from using nuclear fuel. The research and development of such technologies
need to be deliberate. \gls{NFC} transition scenario simulation and 
analysis is the first step in identifying technological needs
and goals. The identified technological needs will drive national
policy and \gls{RD} funding. As demonstrated by this thesis, \Cyclus, with its modular structure and expandable nature, is essential in that effort.

