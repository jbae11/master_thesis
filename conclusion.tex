This thesis demonstrates the capability of Cyclus, the agent-based
fuel cycle simulator, to model real-world nuclear fuel cycle
transition scenarios. Cyclus has a noble framework that is
modular and expandable, that allows addition of functionalities
without editing Cyclus itself.

I developed \texttt{write\_input.py}
to integrate historical nuclear operations by generating
a Cyclus input file from a \gls{PRIS} database. I also
developed an \gls{MSR} module in Cyclus that uses a database
generated by a high-fidelity \gls{MSR} simulation to model
\glspl{MSR} in a large fuel cycle simulation. The two
added capabilities leverage Cyclus' existing capabilities
to model complex, real-world nuclear fuel cycle transition
scenarios.

The Cyclus benchmark study compared Cyclus' results with
the results from other fuel cycle simulators
(
DYMOND \cite{yacout_modeling_2005},
VISION \cite{jacobson_verifiable_2010},
ORION \cite{gregg_analysis_2012}, and
MARKAL \cite{shay_epa_2006}
) for a generic \gls{NFC} transition scenario from an
\gls{LWR} fleet to an \gls{SFR} fleet. Results show
excellent agreement, except for disagreement caused 
by differences in reactor depletion
behavior at the end of reactor lifetime, and
whether fuel discharge is discrete or continuous.

The France \gls{NFC} transition simulation demonstrated
Cyclus' capacity to model the historical nuclear operation
of multiple regions. It also showed that
France can transition from its \gls{LWR} fleet into a fully \gls{SFR} fleet
without additional construction of \glspl{LWR} if France receives
\gls{LWR} \gls{UNF} from other \gls{EU} nations.

The U.S. \gls{NFC} transition simulation demonstrated
Cyclus' capacity to roughly model \glspl{MSR} in
a large \gls{NFC} simulation. The simulation showed
that the U.S. can easily transition into a fully
\gls{MSR} fleet, and that availability of nuclear materials
(\gls{LWR} \gls{UNF} and depleted uranium) is not a constraint.



\section{Future Work}
Continued research into better methods of modeling fuel cycle
transition scenarios and fuel cycle facilities can progress
in multiple directions. 

First, efforts can be made to incorporate
uncertainties in the fuel cycle simulation by adding functionalities
in Cyclus where the parameters are governed by an equation
or sampled from a distribution. Parameters such as \gls{LWR} fuel
burnup and enrichment evolve in time, and can be better modeled
by a function of time than a static parameter. Similarly, reactor parameters
such as refuel time and cycle time can be better described by sampling
from a distribution (e.g. gaussian) rather than the current static
value.

Second, better depletion calculation models can be developed to increase
fuel depletion resolution in reactors. The currently used method for
\glspl{LWR} and \glspl{SFR} are batch-wise, meaning that an average
depleted composition is used for the entire batch, while in reality,
assemblies in a batch do not have the same burnup.

Third, improvements on the \gls{MSR} model can be made so that it
takes into account the varying initial fuel composition. Due to delays
in reprocessing, fuel fabrication, and deployment of \glspl{MSR}, \gls{MSR}
initial fuels contain varying \gls{TRU} vectors. This should be taken into
account by having a database that contains lifetime depletion calculations
of varying initial fuel compositions. Alternatively, an in-simulator
depletion module can be developed (if computational resources allow).


\section{Closing Remarks}
The \gls{ANS} chose closing the \gls{NFC} as its grand challenge. \gls{ANS}
proposes to do so by ``firmly establish the pathway that leads to closing
the nuclear fuel cycle to support the demonstration and deployment of
advanced fission reactors, accelerators, and material recycling technologies
to obtain maximum value while minimizing environmental impact 
from using nuclear fuel.'' The research and development of such technologies
need to be deliberate. \gls{NFC} transition scenario simulation and 
analysis is the first step in identifying technological needs
and goals.  As demonstrated by this thesis, Cyclus, with its
modular structure and expandable nature, is essential in that effort.