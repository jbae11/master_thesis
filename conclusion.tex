
\section{Limitations of the database approach}
The limitations of this database approach is that it does not
take into account the changing incoming fuel compositions due to decay.
The separated \gls{TRU} composition may vary depending on the time
a \gls{LWR} \gls{UNF} has been cooled, thus affecting the performance
of the \gls{MSR}. The database approach assumes a fixed input salt
composition, which is not the case in this simulation, for reprocessed
\gls{TRU} spends varying amounts of time until it is fabricated and
put in the \gls{MSR}.

A way to overcome this limitation is to use a database of databases 
that contain Saltproc simulation results with varying initial \gls{TRU}
vectors. When the reactor archetype receives the fuel, it will find the
Saltproc simulation result that used initial fuel closest to that
of the received fuel salt, minimizing error stemming from varying
\gls{TRU}.

This limitation is similar to that of batch-wise recipe reactors, where the 
pre-generated recipe already has a notion of the composition of initial fuel.
This results in a depleted fuel composition that is agnostic to the incoming
fuel composition in the simulation. For example, a batch-wise recipe reactor
depleting a \gls{MOX} fuel would depleted the \gls{MOX} fuel to a same
composition regardless of its plutonium vector. This is mediated by using
fuel fabrication facility that modifies the plutonium enrichment to
meet a certain fissile value (in Cyclus, the Cycamore \texttt{FuelFab} archetype).

The same method can be applied, where, instead of using a fixed mass ratio
to fabricate fuel for \glspl{MSR}, the fuel fabrication is done by modifying
the \gls{TRU} enrichment to match a certain fissile value. However, this is
not a solution to solve the accuracy problem for depletion and errors
in neutronics calculations.

The best solution is to have a built-in depletion calculation model. However,
as mentioned before, this requires too much computational burden. Another
interesting method is to implement a reduced-order-model of \gls{MSR} depletion
behavior created by training from a large dataset of Saltproc simulations. This
way the computationally heavy work can be outsourced, and the variations
in the initial fuel composition can be taken into account.