This thesis demonstrates the capability of Cyclus, the agent-based
fuel cycle simulator, to model real-world nuclear fuel cycle
transition scenarios. Cyclus has a noble framework that is
modular and expandable, which allows leveraging its capabilities
possible.
I developed \texttt{write\_input.py}
to integrate historical nuclear operations by generating
a Cyclus input file from a \gls{PRIS} database. I also
developed an \gls{MSR} module in Cyclus that uses a database
generated by a high-fidelity \gls{MSR} simulation to model
an \gls{MSR} in a large fuel cycle simulation. The two
added capabilities leverage Cyclus' existing capabilities
to model complex, real-world nuclear fuel cycle transition
scenarios.

The Cyclus benchmark study compared Cyclus' results with
the results from other fuel cycle simulators
(
DYMOND \cite{yacout_modeling_2005},
VISION \cite{jacobson_verifiable_2010},
ORION \cite{gregg_analysis_2012}, and
MARKAL \cite{shay_epa_2006}
) for a generic \gls{NFC} transition scenario from a
\gls{LWR} fleet to a \gls{SFR} fleet. Results show
excellent agreement, except for disagreement caused 
by differences in reactor depletion
behavior at the end of reactor lifetime, and
whether fuel discharge is discrete or continuous.

The France \gls{NFC} transition simulation demonstrated
Cyclus' capacity to model multiple regions' historical
nuclear operation. It also showed that
France can transition from its \gls{LWR} fleet into a fully \gls{SFR} fleet
without additional construction of \glspl{LWR} if France receives
\gls{LWR} \gls{UNF} from other \gls{EU} nations.

The U.S. \gls{NFC} transition simulation demonstrated
Cyclus' capacity to roughly model \glspl{MSR} in
a large \gls{NFC} simulation. The simulation showed
that the U.S. can easily transition into a fully
\gls{MSR} fleet, and that availability of nuclear materials
(\gls{LWR} \gls{UNF}) is not a constraint.



\section{Future Work}
