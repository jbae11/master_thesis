I simulated two different real-world nuclear fuel cycle transition scenarios,
where a \gls{LWR} fleet transitions into a \gls{FR} fleet with 
continuous plutonium and uranium reprocessing. Results show that important
parameters depend heavily on the initial condition of the simulation. France,
due to their long history of reprocessing \gls{LWR} \gls{UNF}, lacked the \gls{LWR}
\gls{UNF} inventory to readily transition to a \gls{SFR} fleet, thus requiring \gls{LWR}
\gls{UNF} from other \gls{EU} nations as well as a \gls{SFR} design with a high
breeding ratio. The U.S. on the other hand, started with a large \gls{LWR} \gls{UNF}
inventory of $68,071$ MTHM (~$7,000$ tons of plutonium), which allowed it the option of an aggressive nuclear capacity growth
and a low-breeding-ratio \gls{SFR} design.

This thesis demonstrates that fuel cycle analysis must be calculated with the real world initial conditions
in consideration, since the initial \gls{UNF} or plutonium inventory can have a large impact on what
nuclear reactor fuel cycle the nation can afford, or what reactor design to use to minimize
waste. In other words, a static, theoretical
fuel cycle analysis with no initial condition is inadequate for providing real-world insight.

This being said, for each fuel cycle analysis,
a central effort needs to be made on the knowledge of a nation's
initial conditions (current \gls{UNF} inventory and composition), its available
technologies (reactor technologies, fuel processing technologies), and the
projected energy demand.