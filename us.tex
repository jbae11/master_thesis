
The United States have been the forerunner of nuclear energy, with a current
installed capacity of about 100 GWe. With its size and long history of nuclear
energy, the United States have accumulated about $70,000$ \gls{MTHM} of \gls{UNF}.
The United States' historical nuclear operation and inventory is tracked using
\Cyclus and is compared with benchmarked results from \gls{ORION} and the
\gls{UNF-STANDARDS} database. The \gls{UNF-STANDARDS} database is a comprehensive,
controlled source of \gls{UNF} information, including dry cask attributes, assembly
data, and economic attributes \cite{peterson_unf-st&dards_2017}. With the successful benchmark,
the simulation is extrapolated to the future. I run two separate simulations, 
one with transition into a `closed' fuel cycle, and the other once-through.

The problem with modeling the U.S. transition scenario is that the U.S. does not have
a defined advanced reactor, whereas France has a central plan to transition into \glspl{ASTRID} \cite{boullis_french_2015, varaine_pre-conceptual_2012}.
Although the most prominent and `canonical' reactor design when considering
transition is the \gls{SFR}, the fact that the U.S. nuclear reactor fleet
is decided by economic interests (industries), this allows me to explore
different options, such as a mixed advanced reactor fleet. 

This chapter includes the results and comparison of two simulations:
\begin{enumerate}
    \item The United States transitions into a `closed' fuel cycle with breeder reactors and reprocessing.
    \item The United States maintains once-through cycle.
\end{enumerate}
