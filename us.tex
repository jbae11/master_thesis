
The United States have been the forerunner of nuclear energy, with a current
installed capacity of about 100 GWe. With its size and long history of nuclear
energy, the United States have accumulated about $70,000$ \gls{MTHM} of \gls{UNF}.

The problem with modeling the U.S. transition scenario is that the U.S. does not have
a defined advanced reactor, whereas France has a central plan to transition into \glspl{ASTRID} \cite{boullis_french_2015, varaine_pre-conceptual_2012}.
Although the most prominent and `canonical' reactor design when considering
transition into fast-spectrum, breeding reactors ]is the \gls{SFR},
the fact that the U.S. nuclear reactor fleet
is decided by economic interests (industries), this allows me to explore
different options, such as the \gls{MSR} design.

\glspl{MSR} reactor designs have recently gained attention in the U.S. due to it being a
potentially safer, more efficient, and sustainable form of nuclear power
\cite{serp_molten_2014}. Multiple companies in the U.S. are now pursuing
commercialization of \gls{MSR} design reactors, such as Tranasatomic \cite{transatomic_technical_2016}
, Terrapower, Terrestrial \cite{leblanc_18-_2017}, and
Thorcon \cite{dolan_19-_2017}. Other parties such as China (TMSR-LF \cite{dai_17-_2017}) 
and the European Union (MSFR \cite{heuer_towards_2014}, MOSART \cite{ignatiev_molten_2014})
are developing \gls{MSR} designs.

In this chapter, I explore the U.S. transition scenario
from a \gls{LWR} fleet into a \gls{MSR} fleet.

\section{Initial Conditions and Scenario Parameters}
Unlike the French scenario,
where the \gls{UNF} inventory at the present time is unknown, there is a
detailed database that describes the U.S. \gls{UNF} inventory up to 2013 May.
The \gls{UNF-STANDARDS} database is a comprehensive,
controlled source of \gls{UNF} information, including dry cask attributes, assembly
data, and economic attributes \cite{peterson_unf-st&dards_2017}. This database
allows the transition scenario simulation to start from 2013, instead of 1970,
like the French simulation. The \gls{UNF} inventory mass and composition in 2013
will be imported from \gls{UNF-STANDARDS} and will be `initiated' in the simulation
as a \texttt{Source} facility. 

\section{U.S. Deployment Schedule}

\subsection{Energy Demand Prediction}

\section{Scenario Specification}

\section{Reactor Specifications}

\section{Material Definitions}

\section{Results}

\section{Conclusion}