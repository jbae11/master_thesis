The goal of this paper is to identify and quantify the
importance of key parameters in fuel cycle transition scenarios
and their impacts on important fuel cycle transition metrics.

Numerous work similar to this nature exists, but they provide
static analyses and does not fully capture the dynamic nature
of the nuclear fuel cycle transition scenario. Also, few
transition scenarios consider real-life scenarios
where an existing \gls{UNF} inventory can be utilize to
accelerate transition.

By running transition scenarios with current \gls{UNF}
inventory and nuclear reactor fleet in consideration,
and running a sensitivity study within the real-world simulation,
this study provides a more realistic, quantitative insight
into what parameters are of importance.

I follow several steps to accomplish this goal.

First, I benchmark \Cyclus, the fuel cycle simulator that is used for this study,
against other fuel cycle simulators. A collective effort has been made previously
by national lab researchers to compare
and benchmark fuel cycle simulators
DYMOND \cite{yacout_modeling_2005},
VISION \cite{jacobson_verifiable_2010},
ORION \cite{gregg_analysis_2012}, and
MARKAL \cite{shay_epa_2006} in a verification study \cite{feng_standardized_2016}.
I replicate the results from the verification study and reveal that \Cyclus'
results agree with other fuel cycle codes with minor differences rising from
reactor module behavior.

Then, key metrics for fuel cycle transition
are identified through literature review and discussion. I then
extract important metrics in the perspective of the general public.

Then, I identify important fuel cycle parameters by performing a parameter
sweep on two real scenarios to visualize the sensitivity of the metrics
for each parameter. With the results, I draw a conclusion on what parameters
are important, and what the future efforts in `closing' the fuel cycle
should focus on.
