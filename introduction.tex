
The goal of this paper is to identify and quantify the
importance of key parameters in fuel cycle transition scenarios
and their impacts on important fuel cycle transition metrics.
I follow several steps to accomplish this goal.

First, I benchmark \Cyclus, the fuel cycle simulator that is used for this study,
against \gls{ORION} \cite{gregg_analysis_2012} with the \gls{UNF-STANDARDS} database
\cite{peterson_used_2013}. The \gls{UNF-STANDARDS} database contains detailed
data on the U.S. legacy nuclear fuel assemblies, and contains the
assemblies' burnup and composition. This allows me to check the fidelity of \Cyclus with a previously-benchmarked code like \gls{ORION}.

Then, key metrics for fuel cycle transition
is identified through literature review and discussion. I then
extract important metrics in the perspective of the general public.

Then, I identify important fuel cycle parameters by performing a parameter
sweep on two real scenarios to visualize the sensitivity of the parameters
on the key metrics.

In a glance, this study sounds very similar to the OECD-NEA study in 2015
by Hyland et al. \cite{hyland_effects_2015}, but this study provides
unique insight in that it explores the sensitivity of incentives to
transition, not just fuel cycle metrics.

Also, this study attempts to apply the similar sensitivity study in
real-world regional scenarios, such as France and the U.S. Note that the purpose
of this study is to identify important parameters, not to make a suggestion
as to what any government should do.
