The scope of this work includes development of
various methods and tools to leverage Cyclus' existing
capabilities to model real-world fuel cycle transition scenarios.

\section{Background and motivation}
Increasing climate change concerns has directed attention
to nuclear energy, which produces reliable base load energy
with negligible CO$_2$ emission. With the reduction of fossil
fuel based power plants and the general increase in energy demand
(28\% growth between 2015 and 2040 \cite{conti_international_2016}),
nuclear power is expected to play a crucial role in the world energy portfolio.

However, concerns of the accumulating \gls{UNF} inventory,
safety of the current reactor fleet, and the availability of
uranium resources create a negative public perception of
nuclear energy and its sustainability.

Advanced fuel cycles utilize reprocessing of the \gls{UNF}
and advanced reactor technology, such as fast-spectrum reactors,
to address those concerns. Reductions by a factor of four
in \gls{HLW} volume can be achieved with reprocessing \cite{widder_benefits_2010},
and advanced reactors provide dramatic improvements "in the four
broad areas of sustainability, economics, safety and reliability,
and proliferation resistance and physical protection" \cite{committee_technology_2002}.

Advanced fuel cycles are much more complex than
the current once-through cycle. The once-through cycle is when a natural uranium source is
enriched and fabricated for use in \glspl{LWR}, and the \gls{UNF} is stored and disposed.
On the other hand, advanced fuel cycles recycle \gls{UNF} from \glspl{LWR} and
fabricate fuel for advanced reactors, such as \glspl{SFR} and \glspl{MSR}. 
The two cycles are shown in figure []. A fuel cycle simulation tool needs
to have the capability to model all the facilities involved, in order to
calculate the material flow and nuclear inventory correctly.

The development of \gls{NFCS}
to simulate transition scenarios
is increasingly becoming important, as multiple nations
consider transitioning out of their \gls{LWR} based fleet
to an advanced \gls{NFC} which utilized reprocessing
of \gls{UNF} and advanced nuclear reactor technologies.

For a successful simulation of a real-world fuel cycle transition
scenario, a fuel cycle simulation tool must be able to 
model the past and current fleet accurately and model advanced
fuel cycle technologies, such as \glspl{MSR} and reprocessing plants.

\section{Objectives}

This thesis demonstrates real-world \gls{NFC} transition
scenario modeling capabilities of Cyclus. The first objective is
to benchmark Cyclus to other fuel cycle simulation tools to
show that Cyclus' results are in good agreement with the results
from other codes. The second objective is to develop tools
necessary to model real-world \gls{NFC} transition
scenarios. The tools include a script that creates a Cyclus
input file of the historical nuclear reactor operation from
a database, and a module that models \gls{MSR} behavior in Cyclus.
Finally, I use Cyclus to construct a simulation of the nuclear
fuel cycle transition scenario for France and the United States,
to obtain metrics such as \gls{UNF} and raffinate inventory.


\section{Methods}
This thesis accomplishes the objective in three steps. First,
the general fuel cycle simulation tool, Cyclus, is benchmarked
to demonstrate its agreement with other fuel cycle simulation
tools. Second, I identify the tools and methods necessary
for modeling and simulating real-world transition scenarios.
Finally, I construct and run fuel cycle transition scenarios
for France and the United States.

A previous study \cite{feng_standardized_2016} validates existing fuel cycle
simulation tools in a fuel cycle transition scenario, where a \gls{LWR} fleet
transitions into an \gls{SFR} fleet with continuous reprocessing. This 
study compares four well-known \glspl{NFCS}
DYMOND \cite{yacout_modeling_2005},
VISION \cite{jacobson_verifiable_2010},
ORION \cite{gregg_analysis_2012}, and
MARKAL \cite{shay_epa_2006}. The results from each code were
compared to a set of `model solutions' that were generated
from an excel worksheet for different metrics (e.g. fuel loading
in reactor, \gls{UNF} inventory). I reproduce the transition
scenario on Cyclus, and compare the Cyclus results with those
from the `model solutions'.

In order to model real-world transition scenarios into an advanced
fuel cycle, I developed two major tools. First, I developed a python
module that automates extraction from the curated \gls{IAEA} \gls{PRIS} database
\cite{iaea_nuclear_2018}. The database lists each nuclear reactor's
country, name, type, net capacity (\gls{MWe}), status, operator, construction
date, first criticality date, first grid date, commercial date, and shutdown
date (if applicable). The module extracts the information from this file
to generate a Cyclus compatible input file, which lists the individual
reactor units as agents. Second, I developed a tool that models \glspl{MSR}
using a database generated from a high-fidelity \gls{MSR} depletion calculation.
The database is an output of Saltproc [!!! Cite saltproc], which is a python
module that drives
SERPENT 2 \cite{leppanen_serpentcontinuous-energy_2013} to model on-line reprocessing in an \gls{MSR}.
The hdf5 database contains isotopic timeseries of streams in and out of the reactor,
isotopic timeseries inside the reactor, and keff values. The developed tool then
reads the hdf5 file to mimic \gls{MSR} behavior by requesting and offering
material that is listed in the database.

Finally I construct the fuel cycle transition scenario for France and the United States.
I make different assumptions for the two scenarios due to each nation's different goals,
initial conditions (currently existing fleet, \gls{UNF} inventory), and potential reactor
technology.

The structure of this thesis is as follows. In chapter 2, I review other fuel cycle simulation
tools and their gaps, and explain why Cyclus
has the unique capability for modeling real-world fuel cycle transition scenarios.
In chapter 3, I list and justify my assumption for the fuel cycle transition
scenario for France and the United States.
Chapter 4 shows the results from the benchmark study, where Cyclus results are compared
to results from other fuel cycle simulation tools.
Chapter 5 contains explanation and demonstration of  
 I list and justify my assumptions
for the fuel cycle transition scenario for France and the United States. 

\iffalse
For France, I model the entire \gls{EU} region to calculate \gls{UNF} inventory
in each \gls{EU} nation. This is because France would need to receive \gls{UNF} from other
nations to quickly transition into a \gls{SFR} fleet, since their \gls{UNF} inventory
is small due to their long history of reprocessing for \gls{LWR} \gls{MOX} fuel production.
\fi

