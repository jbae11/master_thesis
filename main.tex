
%% Package and Class "uiucthesis2014" for use with LaTeX2e.
\documentclass{article}

\usepackage[acronym,toc]{glossaries}
%\newacronym{<++>}{<++>}{<++>}
\newacronym[longplural={metric tons of heavy metal}]{MTHM}{MTHM}{metric ton of heavy metal}
\newacronym{ABM}{ABM}{agent-based modeling}
\newacronym{ACDIS}{ACDIS}{Program in Arms Control \& Domestic and International Security}
\newacronym{ADS}{ADS}{Accelerator-Driven Systems}
\newacronym{AHTR}{AHTR}{Advanced High Temperature Reactor}
\newacronym{ANDRA}{ANDRA}{Agence Nationale pour la gestion des D\'echets RAdioactifs, the French National Agency for Radioactive Waste Management}
\newacronym{ANL}{ANL}{Argonne National Laboratory}
\newacronym{ANS}{ANS}{American Nuclear Society}
\newacronym{API}{API}{application programming interface}
\newacronym{ARE}{ARE}{Aircraft Reactor Experiment}
\newacronym{ARFC}{ARFC}{Advanced Reactors and Fuel Cycles}
\newacronym{ASME}{ASME}{American Society of Mechanical Engineers}
\newacronym{ASTRID}{ASTRID}{Advanced Sodium Technological Reactor for Industrial Demonstration}
\newacronym{ATWS}{ATWS}{Anticipated Transient Without Scram}
\newacronym{BDBE}{BDBE}{Beyond Design Basis Event}
\newacronym{BIDS}{BIDS}{Berkeley Institute for Data Science}
\newacronym{BWR}{BWR}{Boiling Water Reactor}
\newacronym{CAFCA}{CAFCA}{ Code for Advanced Fuel Cycles Assessment }
\newacronym{CANDU}{CANDU}{Canada Deuterium Uranium}
\newacronym{CDTN}{CDTN}{Centro de Desenvolvimento da Tecnologia Nuclear}
\newacronym{CEA}{CEA}{Commissariat \`a l'\'Energie Atomique et aux \'Energies Alternatives}
\newacronym{CI}{CI}{continuous integration}
\newacronym{CNEN}{CNEN}{Comiss\~{a}o Nacional de Energia Nuclear}
\newacronym{CNERG}{CNERG}{Computational Nuclear Engineering Research Group}
\newacronym{CORRM}{CORRM}{Continuous On-Line Reprocessing Reactor Module}
\newacronym{COSI}{COSI}{Commelini-Sicard}
\newacronym{COTS}{COTS}{commercial, off-the-shelf}
\newacronym{CSNF}{CSNF}{commercial spent nuclear fuel}
\newacronym{CTAH}{CTAHs}{Coiled Tube Air Heaters}
\newacronym{CUBIT}{CUBIT}{CUBIT Geometry and Mesh Generation Toolkit}
\newacronym{CURIE}{CURIE}{Centralized Used Fuel Resource for Information Exchange}
\newacronym{DAG}{DAG}{directed acyclic graph}
\newacronym{DANESS}{DANESS}{Dynamic Analysis of Nuclear Energy System Strategies}
\newacronym{DBE}{DBE}{Design Basis Event}
\newacronym{DESAE}{DESAE}{Dynamic Analysis of Nuclear Energy Systems Strategies}
\newacronym{DHS}{DHS}{Department of Homeland Security}
\newacronym{DOE}{DOE}{Department of Energy}
\newacronym{DU}{DU}{depleted uranium}
\newacronym{DRACS}{DRACS}{Direct Reactor Auxiliary Cooling System}
\newacronym{DRE}{DRE}{dynamic resource exchange}
\newacronym{DSNF}{DSNF}{DOE spent nuclear fuel}
\newacronym{DYMOND}{DYMOND}{Dynamic Model of Nuclear Development }
\newacronym{EBS}{EBS}{Engineered Barrier System}
\newacronym{EDF}{EDF}{Électricité de France}
\newacronym{EFPD}{EFPD}{Effective Full Power Days}
\newacronym{EDS}{EDS}{Externally Driven Systems}
\newacronym{EDZ}{EDZ}{Excavation Disturbed Zone}
\newacronym{EIA}{EIA}{U.S. Energy Information Administration}
\newacronym{EPA}{EPA}{Environmental Protection Agency}
\newacronym{EPR}{EPR}{European Pressurized Reactor}
\newacronym{EP}{EP}{Engineering Physics}
\newacronym{EU}{EU}{European Union}
\newacronym{FCO}{FCO}{Fuel Cycle Options}
\newacronym{FCT}{FCT}{Fuel Cycle Technology}
\newacronym{FEHM}{FEHM}{Finite Element Heat and Mass Transfer}
\newacronym{FEPs}{FEPs}{Features, Events, and Processes}
\newacronym{FHR}{FHR}{Fluoride-Salt-Cooled High-Temperature Reactor}
\newacronym{FLiBe}{FLiBe}{Fluoride-Lithium-Beryllium}
\newacronym{FP}{FP}{Fission Product}
\newacronym{GDSE}{GDSE}{Generic Disposal System Environment}
\newacronym{GDSM}{GDSM}{Generic Disposal System Model}
\newacronym{GENIUSv1}{GENIUSv1}{Global Evaluation of Nuclear Infrastructure Utilization Scenarios, Version 1}
\newacronym{GENIUSv2}{GENIUSv2}{Global Evaluation of Nuclear Infrastructure Utilization Scenarios, Version 2}
\newacronym{GENIUS}{GENIUS}{Global Evaluation of Nuclear Infrastructure Utilization Scenarios}
\newacronym{GPAM}{GPAM}{Generic Performance Assessment Model}
\newacronym{GRSAC}{GRSAC}{Graphite Reactor Severe Accident Code}
\newacronym{GUI}{GUI}{graphical user interface}
\newacronym{HEU}{HEU}{high enriched uranium}
\newacronym{HLW}{HLW}{high level waste}
\newacronym{HPC}{HPC}{high-performance computing}
\newacronym{HTC}{HTC}{high-throughput computing}
\newacronym{HWR}{HWR}{Heavy Water Reactor}
\newacronym{HTGR}{HTGR}{High Temperature Gas-Cooled Reactor}
\newacronym{IAEA}{IAEA}{International Atomic Energy Agency}
\newacronym{IEA}{IEA}{International Energy Agency}
\newacronym{IEMA}{IEMA}{Illinois Emergency Mangament Agency}
\newacronym{IHLRWM}{IHLRWM}{International High Level Radioactive Waste Management}
\newacronym{INL}{INL}{Idaho National Laboratory}
\newacronym{IPRR1}{IRP-R1}{Instituto de Pesquisas Radioativas Reator 1}
\newacronym{IRP}{IRP}{Integrated Research Project}
\newacronym{ISFSI}{ISFSI}{Independent Spent Fuel Storage Installation}
\newacronym{ISRG}{ISRG}{Independent Student Research Group}
\newacronym{JFNK}{JFNK}{Jacobian-Free Newton Krylov}
\newacronym{LANL}{LANL}{Los Alamos National Laboratory}
\newacronym{LBNL}{LBNL}{Lawrence Berkeley National Laboratory}
\newacronym{LCOE}{LCOE}{levelized cost of electricity}
\newacronym{LDRD}{LDRD}{laboratory directed research and development}
\newacronym{LFR}{LFR}{Lead-Cooled Fast Reactor}
\newacronym{LLNL}{LLNL}{Lawrence Livermore National Laboratory}
\newacronym{LLW}{LLW}{Low Level Waste}
\newacronym{LMFBR}{LMFBR}{Liquid Metal Fast Breeder Reactor}
\newacronym{LOFC}{LOFC}{Loss of Forced Cooling}
\newacronym{LOHS}{LOHS}{Loss of Heat Sink}
\newacronym{LOLA}{LOLA}{Loss of Large Area}
\newacronym{LP}{LP}{linear program}
\newacronym{LWR}{LWR}{Light Water Reactor}
\newacronym{MAGNOX}{MAGNOX}{Magnesium Alloy Graphie Moderated Gas Cooled Uranium Oxide Reactor}
\newacronym{MA}{MA}{minor actinide}
\newacronym{MCNP}{MCNP}{Monte Carlo N-Particle code}
\newacronym{MCSFR}{MCSFR}{Molten Chloride Salt Fast Reactor}
\newacronym{MILP}{MILP}{mixed-integer linear program}
\newacronym{MIT}{MIT}{the Massachusetts Institute of Technology}
\newacronym{MOAB}{MOAB}{Mesh-Oriented datABase}
\newacronym{MOOSE}{MOOSE}{Multiphysics Object-Oriented Simulation Environment}
\newacronym{MOX}{MOX}{Mixed Oxide Fuel}
\newacronym{MSBR}{MSBR}{Molten Salt Breeder Reactor}
\newacronym{MSRE}{MSRE}{Molten Salt Reactor Experiment}
\newacronym{MSR}{MSR}{Molten Salt Reactor}
\newacronym{MWe}{MWe}{Megawatt Electric}
\newacronym{NAGRA}{NAGRA}{National Cooperative for the Disposal of Radioactive Waste}
\newacronym{NEAMS}{NEAMS}{Nuclear Engineering Advanced Modeling and Simulation}
\newacronym{NEUP}{NEUP}{Nuclear Energy University Programs}
\newacronym{NFC}{NFC}{nuclear fuel cycle}
\newacronym{NFCS}{NFC simulator}{nuclear fuel cycle simulator}
\newacronym{NGNP}{NGNP}{Next Generation Nuclear Plant}
\newacronym{NMWPC}{NMWPC}{Nuclear MW Per Capita}
\newacronym{NNSA}{NNSA}{National Nuclear Security Administration}
\newacronym{NPRE}{NPRE}{Department of Nuclear, Plasma, and Radiological Engineering}
\newacronym{NQA1}{NQA-1}{Nuclear Quality Assurance - 1}
\newacronym{NRC}{NRC}{Nuclear Regulatory Commission}
\newacronym{NSF}{NSF}{National Science Foundation}
\newacronym{NSSC}{NSSC}{Nuclear Science and Security Consortium}
\newacronym{NUWASTE}{NUWASTE}{Nuclear Waste Assessment System for Technical Evaluation}
\newacronym{NWF}{NWF}{Nuclear Waste Fund}
\newacronym{NWTRB}{NWTRB}{Nuclear Waste Technical Review Board}
\newacronym{OCRWM}{OCRWM}{Office of Civilian Radioactive Waste Management}
\newacronym{ORION}{ORION}{ORION}
\newacronym{ORNL}{ORNL}{Oak Ridge National Laboratory}
\newacronym{PARCS}{PARCS}{Purdue Advanced Reactor Core Simulator}
\newacronym{PBAHTR}{PB-AHTR}{Pebble Bed Advanced High Temperature Reactor}
\newacronym{PBFHR}{PB-FHR}{Pebble-Bed Fluoride-Salt-Cooled High-Temperature Reactor}
\newacronym{PEI}{PEI}{Peak Environmental Impact}
\newacronym{PH}{PRONGHORN}{PRONGHORN}
\newacronym{PHWR}{PHWR}{Pressurized Heavy Water Reactor}
\newacronym{PRIS}{PRIS}{Power Reactor Information System}
\newacronym{PRKE}{PRKE}{Point Reactor Kinetics Equations}
\newacronym{PSPG}{PSPG}{Pressure-Stabilizing/Petrov-Galerkin}
\newacronym{PWAR}{PWAR}{Pratt and Whitney Aircraft Reactor}
\newacronym{PWR}{PWR}{Pressurized Water Reactor}
\newacronym{PyNE}{PyNE}{Python toolkit for Nuclear Engineering}
\newacronym{PyRK}{PyRK}{Python for Reactor Kinetics}
\newacronym{QA}{QA}{quality assurance}
\newacronym{RDD}{RD\&D}{Research Development and Demonstration}
\newacronym{RD}{R\&D}{Research and Development}
\newacronym{RELAP}{RELAP}{Reactor Excursion and Leak Analysis Program}
\newacronym{RIA}{RIA}{Reactivity Insertion Accident}
\newacronym{RIF}{RIF}{Region-Institution-Facility}
\newacronym{RU}{RU}{reprocessed uranium}
\newacronym{ROM}{ROM}{Reduced Order Model}
\newacronym{SFR}{SFR}{Sodium-Cooled Fast Reactor}
\newacronym{SINDAG}{SINDA{\textbackslash}G}{Systems Improved Numerical Differencing Analyzer $\backslash$ Gaski}
\newacronym{SKB}{SKB}{Svensk K\"{a}rnbr\"{a}nslehantering AB}
\newacronym{SNF}{SNF}{spent nuclear fuel}
\newacronym{SNL}{SNL}{Sandia National Laboratory}
\newacronym{STC}{STC}{specific temperature change}
\newacronym{SUPG}{SUPG}{Streamline-Upwind/Petrov-Galerkin}
\newacronym{SWF}{SWF}{Separations and Waste Forms}
\newacronym{SWU}{SWU}{Separative Work Unit}
\newacronym{TRIGA}{TRIGA}{Training Research Isotope General Atomic}
\newacronym{TRISO}{TRISO}{Tristructural Isotropic}
\newacronym{TRU}{TRU}{transuranic}
\newacronym{TSM}{TSM}{Total System Model}
\newacronym{TSPA}{TSPA}{Total System Performance Assessment for the Yucca Mountain License Application}
\newacronym{ThOX}{ThOX}{thorium oxide}
\newacronym{UDB}{UDB}{Unified Database}
\newacronym{UFD}{UFD}{Used Fuel Disposition}
\newacronym{UML}{UML}{Unified Modeling Language}
\newacronym{UNF}{UNF}{Used Nuclear Fuel}
\newacronym{UNF-STANDARDS}{UNF-ST\&DARDS}{Used Nuclear Fuel Storage Transportation and Disposal Analysis Resource and Data System}
\newacronym{UOX}{UOX}{Uranium Oxide Fuel}
\newacronym{UQ}{UQ}{uncertainty quantification}
\newacronym{US}{US}{United States}
\newacronym{UW}{UW}{University of Wisconsin}
\newacronym{VISION}{VISION}{the Verifiable Fuel Cycle Simulation Model}
\newacronym{VVER}{VVER}{Voda-Vodyanoi Energetichesky Reaktor (Russian Pressurized Water Reactor)}
\newacronym{VV}{V\&V}{verification and validation}
\newacronym{WIPP}{WIPP}{Waste Isolation Pilot Plant}
\newacronym{YMR}{YMR}{Yucca Mountain Repository Site}

\usepackage{placeins}
\usepackage{booktabs} % nice rules (thick lines) for tables
\usepackage{microtype} % improves typography for PDF
\usepackage{xspace}
\usepackage[hidelinks]{hyperref}
\usepackage{xspace}
\usepackage{hhline}
\usepackage{amsmath}
\usepackage{color}
\usepackage{multirow}
\usepackage{fourier}
\usepackage{booktabs}
\usepackage{threeparttable, tablefootnote}

%tikzpicture fit to page width
\usepackage{environ}
\makeatletter
\newsavebox{\measure@tikzpicture}
\NewEnviron{scaletikzpicturetowidth}[1]{%
  \def\tikz@width{#1}%
  \def\tikzscale{1}\begin{lrbox}{\measure@tikzpicture}%
  \BODY
  \end{lrbox}%
  \pgfmathparse{#1/\wd\measure@tikzpicture}%
  \edef\tikzscale{\pgfmathresult}%
  \BODY
}

\usepackage{tabularx}
\newcolumntype{b}{>{\hsize=1.0\hsize}X}
\newcolumntype{s}{>{\hsize=.5\hsize}X}
\newcolumntype{m}{>{\hsize=.75\hsize}X}
\usepackage{graphics}
\newcommand{\Cyclus}{\textsc{Cyclus}\xspace}%
\newcommand{\uthree}{\xspace $^{233}_{92}U$ \xspace}
\newcommand{\ufive}{\xspace $^{235}_{92}U$\xspace}%%
\newcommand{\utwo}{\xspace $^{232}_{92}U$\xspace}
\newcommand{\ueight}{\xspace $^{238}_{92}U$\xspace}
\newcommand{\pu}{\xspace$^{239}_{94}Pu$\xspace}
\newcommand{\thor}{\xspace $^{232}_{90}Th$\xspace}
\graphicspath{ {images/} }
\usepackage[affil-it]{authblk}
\usepackage[numbers]{natbib}
\usepackage{notoccite}
\usepackage{tikz}
\usetikzlibrary{positioning, arrows, decorations, shapes}

\usetikzlibrary{shapes.geometric,arrows}
\tikzstyle{process} = [rectangle, rounded corners, minimum width=2.5cm, minimum height=1cm,text centered, draw=black, fill=blue!30]
\tikzstyle{object} = [ellipse, rounded corners, minimum width=3cm, minimum height=1cm,text centered, draw=black, fill=green!30]
\tikzstyle{empty} =  [rectangle, rounded corners, minimum width=2.5cm, minimum height=0.7cm,text centered, draw=black, fill=white!30]
\tikzstyle{arrow} = [thick,->,>=stealth]
\usepackage{cleveref}
\usepackage{datatool}


\begin{document}

\title{Nuclear Fuel Cycle Transition Scenarios based upon current inventory}
\author{Jin Whan Bae}


%% Create an abstract that can also be used for the ProQuest abstract.
%% Note that ProQuest truncates their abstracts at 350 words.
\begin{abstract}
This work analyses the historic operation of nuclear energy and the \gls{UNF} inventory.
With the calculated inventory, we suggest different
directions of fuel cycle transition scenarios that will maximize the currently
available resource. Several case studies of this includes the European region,
and the United States.
\end{abstract} 

%% Create a dedication in italics with no heading, centered vertically
%% on the page.
%% Create an Acknowledgements page, many departments require you to
%% include funding support in this.
\chapter*{Acknowledgments}


%% The thesis format requires the Table of Contents to come
%% before any other major sections, all of these sections after
%% the Table of Contents must be listed therein (i.e., use \chapter,
%% not \chapter*).  Common sections to have between the Table of
%% Contents and the main text are:
%%
%% List of Tables
%% List of Figures
%% List Symbols and/or Abbreviations
%% etc.

\tableofcontents
\listoftables
\listoffigures

%% Create a List of Abbreviations. The left column
%% is 1 inch wide and left-justified
\chapter{List of Abbreviations}
\printglossaries
%% Create a List of Symbols. The left column
%% is 0.7 inch wide and centered
\chapter{List of Symbols}


\pagebreak

\section{Background}
Nuclear energy is the largest provider of carbon-free energy
in the United States. However, several problems remain with nuclear
energy, with one of the biggest one being the problem of nuclear waste.
From the 1950s, the United States have been trying to commission a 
geological waste disposal site for its `spent' fuel. However, even
after 60 years, this endeavor has yet to accomplish of substantial value.

On the other hand, the recent advances in nuclear technology produced
a multitude of new reactor designs. The new reactors designs also 
experiment with new forms of fuel that deviate from an enriched uranium
oxide fuel. (why? fast spectrum and molten salt) These advances in new
technology has fueled the dream of a `closed' fuel cycle, where little to no
supply of natural material is required, that `used' fuel from one reactor,
after processing, could fuel another reactor.

However, transitioning the nuclear fuel cycle requires great effort,
from constructing new reactors to new fuel processing infrastructure.
Thus, it seems highly unlikely that, given the current circumstances
such as the economic competitiveness of nuclear, uprising of natural
gas, an administration unenthusiastic towards sustainability, that 
this transition will happen soon.
However, the question remains as to what is important when considering
a nuclear fuel cycle transition - what will `nudge' the public to 
support such a great effort \cite{leonard_richard_2008}? What do we
need to focus on to `close' the fuel cycle? What parameters are
important?

There have been many honest efforts by expert groups 
to identify and quantify the key metrics of fuel cycles,
 such as the Evaluation and Screening Study
\cite{wigeland_nuclear_2014} and the OECD-NEA study on uncertainty
\cite{hyland_effects_2015}.



\begin{table}[h]
    \centering
    \caption{Parameters considered in uncertainties study 
             in OECD-NEA study \cite{hyland_effects_2015}.}
    \begin{tabular}{cc}
        \hline
        \textbf{Category} & \textbf{Parameter} \\ \hline
        \multirow{4}{*}{Fissile Inventory and Consumption} & Fissile Burnup \\ 
         & Fresh fuel composition \\ 
         & Cycle length \\ 
         & Breeding Grain\\ \hline
        \multirow{4}{*}{Waste} & Initial \gls{MA} content in fuel \\
         & Recuperation Rate (MA) \\
         & Reprocessing Efficiency \\
         & Enrichment Tail \\ \hline
        \multirow{4}{*}{Fabrication and Reprocessing} & Minimum cooling time \\
        & Fabrication Time\\
        & Reprocessing begin time\\
        & Reprocessing capacity\\ \hline
        \multirow{4}{*}{General} & Reactor Lifetime\\
        & Nuclear Energy Demand \\
        & FR introduction time \\
        & Rate of transition \\ \hline
        
    \end{tabular}
\end{table}




\section{Introduction}

The goal of this paper is to identify and quantify the 
importance of key parameters in fuel cycle transition scenarios.
To accomplish this goal, several steps are followed.

First, \Cyclus, the code that is used for this study, is benchmarked
against ORION \cite{gregg_analysis_2012} and the UNF-ST\&DARDS database
\cite{peterson_used_2013}. The UNF-ST\&DARDS database contains detailed
data on the U.S. legacy nuclear fuel assemblies, and contains the
assemblies' burnup and composition.



This paper analyzes a real scenario, where the fuel cycle
simulation of the historical nuclear operation of a nation is run,
while tracking the inventory. This simulation will provide insight into
the available resources (e.g. plutonium) and the required repository
capacity if the nation maintains the status quo.

After the historical operation to the current date, two different 
scenarios are run, one with transition into a `closed' fuel cycle
and another the status quo. Various metrics of the nuclear fuel cycle,
like accumulated \gls{UNF} mass, decay heat, natural resources used, etc.

Fuel cycle transition scenarios are very uncertain, and these uncertainties
are needed to be taken into account, like done in \cite{nuclear_phathanapirom_2016}. Key uncertainties, such as energy demand growth
rate, technology availability, and costs, should be modeled with uncertainty
in mind. Thus, the basic scenarios are varied with the key parameters
swept over a reasonable parameter range and visualized. 

To do the parameter sweep of fuel cycle transition scenarios without
treacherous labor, we need a demand-driven deployment algorithm,
an algorithm to runs multiple simulations while sweeping over
variables, and an algorithm that allows visualization of 
the parameter sweep simulations. Luckily, 






\section{Method}

\begin{enumerate}
\item Construct basic scenario
\item Do a `base case' of transition scenario of three countries, using published parameters (breeding ratio, reactor specifications, lifetime etc)
\item Parameter sweep on key uncertainties (cost, advanced reactor availability,
energy demand growth rate etc)
\item visualization and discussion
\end{enumerate}


\section{Literature Review}

\subsection{Sustainability Metrics from Nuclear Fuel Cycle Studies}

\subsection{Sustainability Metrics in Other Domains}

\subsection{Differences in U and Th Fuel Cycles}

%\chapter{Conclusions}


%\appendix*
%\include{Appendix.tex}

\section{Fuel Cycles to Compare}

\bibliographystyle{unsrtnat}
\bibliography{bibliography}


\end{document}
\endinput
%%
%% End of file `thesis-ex.tex'.
