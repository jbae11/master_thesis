\glspl{NFCS} are system-level analysis tools
that allow tracking of material flow in a \gls{NFC}. Its
functionalities include, but are not limited to, isotopic decay,
depletion calculations, and separating of material streams.
The goal of a \gls{NFCS} is to calculate `metrics' - quantitative
measures of performance that can be compared among fuel cycle
options \cite{huff_fundamental_2016}.

\section{Capabilities required for modeling transition scenarios}
There are specific capabilities required for modeling transition
scenarios. A study by Brown et al. \cite{brown_identification_2016}
identified nine common functionalities for \glspl{NFCS} for modeling
transition scenarios - material compositions, deployment of fuel
cycle facilities, front-end facility models, separations and material
recycle facilities, reactor facilities, back-end features, starting
the new fuel cycle, materials queuing and prioritization under
capacity limitations, and energy demand algorithms. Brown et al.
categorizes each functionality into three tiers - basic, integral,
and exemplary. The functionalities, features, and their hierarchies
are organized in table \ref{tab:ffh}.

\begin{table}[h]
    \centering
    \caption {Nine common functionalities identified for \gls{NFCS} to perform
    			fuel cycle transition scenarios. Reproduced from Brown et al. \cite{brown_identification_2016}}
    \label{tab:ffh}
    \begin{tabular}{lll}
        \hline
        \textbf{Functionality}&\textbf{Feature} &\textbf{Hierarchy} \\
        \hline
        \multirow{6}{*}{\shortstack{Composition \\ Features}} & \shortstack{Modeling of implicit consideration of fuel materials \\  including primary fissile and fertile actinide isotopes} & Basic \\
        	& \shortstack{Fuel's initial heavy metal mass modeled as lumped masses \\ of the remaining actinides and fission products to conserve mass} & Basic \\
        	& Isotopic decay of materials in storage & Exemplary \\
        	& Modeling of intermediate isotopes (e.g. Pa-233) & Exemplary \\
        	& Tracking of fission products beyond a simple lumped sum & Exemplary \\
        	& Modeling of compounding materials in fuels and waste forms & Exemplary\\

        \hline

        \multirow{3}{*}{\shortstack{Fuel Cycle \\ Facility \\ Deployment}} & Facility deployment and retirement & Basic \\
        	& Construction time delays & Basic \\
        	& Strategic deployment to meet demand &Integral \\

        \hline

        \multirow{6}{*}{\shortstack{Front-end \\ Facilities}} & Source (mining and milling) & Basic \\
        	& Details of mines and mills including annual and total quantities available & Exemplary \\
        	& Conversion and enrichment facilities & Basic \\
        	& Timing and capacity of recycle facilities & Basic \\
        	& Fuel fabrication & Basic \\
        	& Time delays and losses in separations and fabrication & Basic \\

        \hline

        \multirow{4}{*}{\shortstack{Separations \\ and \\ Material \\ Recycling}} & Separations facilities may be required for \gls{UNF} & Basic \\
        	& Cooling time & Basic \\
        	& Losses in separations & Basic \\
        	& Material selection from the \gls{UNF} supply & Basic \\

        \hline

        \multirow{6}{*}{\shortstack{Reactor \\ Facility}} & Fueling: number of batches, cycle length and fuel per batch & Basic \\
        	& Multiple fuel types in reactor facility (driver, blanket) & Basic \\
        	& Pre-generated charge and discharge isotopic compositions & Basic \\
        	& Real time calculations based on reactor physics models & Exemplary \\
        	& Reactor facility lifetime, construction time, and decommissioning time & Basic \\
        	& Initial charge for first core and discharge for final core & Basic \\

        \hline

        \multirow{2}{*}{Back-end} & Cooling of used fuel & Basic \\
        	& \shortstack{Conservation of mass - consistency with\\ charged mass and generated power} & Basic \\

        \hline

        \multirow{3}{*}{\shortstack{Fuel Cycle \\ Startup}} & External source of fissile material & Basic \\
        	& Startup on recycled fuel from other facilities &Integral \\
        	& Primary and back-up fuel types & Exemplary \\

        \hline

        \multirow{3}{*}{\shortstack{Material \\ Prioritization}} & material accumulation & Basic \\
        	& Material prioritization &Integral \\
        	& Radioactive decay & Exemplary \\

        \hline

        \multirow{2}{*}{\shortstack{Energy Demand \\ Algorithm}} & Technology allocation accounting for availability &Integral \\
        	& Ordering and deployment of multiple reactor technologies &Integral \\
        \hline
    \end{tabular}
\end{table}

\section{Additional capabilities identified for real-world fuel cycle transition scenario}
From the functionalities identified in Brown et al., I identified two additional functionalities -
modeling discrete, real-life reactor fleets and operational history; modeling liquid-fueled
reactors with continuous reprocessing.

\subsection{Modeling liquid-fueled reactors with continuous reprocessing}
The reason for needing the liquid-fueled reactor modeling
is due to the recent rise of interest in \glspl{MSR} in the United States. For \glspl{SFR},
which are considered the most prominent fast-spectrum advanced reactor, they behave similarly to \glspl{LWR}
in that they have discrete batches and cycle times. However, \glspl{MSR} have a unique behavior
of continuous refueling and on-line reprocessing which makes the material flow in to and out from the reactor
dynamic in both quantity and composition. This capability requires 



\section{Cyclus}




\begin{itemize}
	\item Required capabilities for modeling transition scenarios
	\begin{itemize}
		\item Real-World modeling
		\item Advanced Reactor modeling
		\item Reprocessing and transmutation
	\end{itemize}
	\item Current state of other \gls{NFCS}
	\item Current State of Cyclus
	\item Gaps in Cyclus that needs to be filled
	\item Tools developed in this paper
\end{itemize}